\chapter{Conclusions and Future Work} % Main chapter title
\label{chap:conclusion} % Change X to a consecutive number; for referencing this chapter elsewhere, use \ref{ChapterX}

\section{Conclusions}
{
Parametric array loudspeakers (PALs) are known for their capability of generating highly
directional audio sound waves. Owing to this feature, they are used in various kinds of audio systems expecting directional sound sources.
For example, they are used as secondary sources in active noise control (ANC) systems to mitigate the unwanted noise in the target regions with minimizing side effects on other areas.
This thesis has investigated the feasibility of using multiple PALs in an ANC system to create a large quiet zone.
Because the physical generation of the audio sound waves from PALs differs from the traditional dynamic loudspeakers,
improved prediction models and physical properties for PALs have been explored.
}

% A systematic investigation on the properties of audio sound generated by a PAL, and the ANC system using PALs as secondary sources is conducted in this thesis.
% The background and motivation of this thesis are presented at the beginning.
% Followed by them is an extensive literature review on the field of PAL and ANC.
% The physical mechanisms and fundamental properties of audio sound generated by a PAL are introduced at first.
% After that, the existing prediction models are summarized, which include the quasilinear approximation, the far field solution, and the governing equations.
% The implementation and applications of PALs, and commercial products are presented.
% The ANC system for the purpose of the generation of a quiet zone is surveyed. 
% Finally, existing work on the ANC systems using directional or PAL loudspeakers are reviewed.

{
    In Sec.~\ref{chap:sound_field}, the commonly used governing equations for PALs are reviewed including the Lighthill equation, second-order nonlinear wave equation, Kuznetsov equation, Westervelt equation, and KZK equation.
    Existing methods utilize the three-dimensional (3D) prediction model, which results in a five-fold integral to solve.
    Section \ref{sec:sound_field_quasi} proposed the framework of a two-dimensional (2D) prediction model.
    Its solution is a three-fold integral which is simpler than the 3D model, so it brings convenience in modelling a PAL.
    Based on the prediction models, it is found in Sec.~\ref{eq:sound_field_back} that the audio sound field on front side of a PAL can  be divided into three regions: the near field, the Westervelt far field, and the inverse-law far field.
    The reason for this division is that the terms \quotes{near field} and \quotes{far field} are usually used in existing literatures without a clear definition.
    This ambiguity may lead to misleading conclusions. 
    For example, the distance of 4 m is sometimes considered as the inverse-law far field of the PAL, so the far field solution was used to predict the audio sound \cite{Shi2015ConvolutionModelComputing}.
    However, it is showed in Sec.~\ref{sec:swe_pal} that 
    the inverse-law far field for a PAL is found to be more than 10 m away from a PAL with a diameter of 0.04 m.
    Therefore, differences between the predictions and measurements continue to be observed \cite{Shi2015ConvolutionModelComputing}.
    % These two terms are currently used for traditional loudspeakers without ambiguity, because the far field usually means the region where the sound pressure amplitude is inversely proportional to the propagation distance.
    % However, it is usually unclear for describing the audio sound field generated by a PAL.
    % Therefore, inaccurate 
% Instead, the Westervelt far field is more useful and applicable in audio applications, which is defined as the region where the Westervelt equation is accurate enough.
    % Simple formula
    With the division proposed in this thesis, appropriate models can now be chosen for different regions to enable faster and more accurate sound field calculation. 
}

{
    There is a concensus among researchers that there should be no audio sound on the back side of a non-baffled PAL due to its sharp directivity on front side.
    % This thesis was the first to investigate the audio sound on back side of a non-baffled PAL, where it is usually considered to be no sound.
    % The audio sound on back side of a PAL is also investigated based on the disk scattering theory.
    This thesis was the first to demonstrate by both simulations and measurements in Sec.~\ref{eq:sound_field_back} that there exists audio sound on back side of a non-baffled PAL.
    This is caused by the diffraction of the demodulated audio waves instead of the ultrasonic waves.
    Therefore, the sound level is larger at lower audio frequency as the diffraction effects are more significant at large wavelengths.
    This phenomenon indicates that the audio sound generated by a PAL is not perfectly unidirectional. 
    The audio sound propagating on back side of the PAL should be taken into account in some applications.
    For example, when the reference sensor in ANC systems is placed on back side of the PAL, the reference signal might be interfered with the feedback noise from the PAL at low frequencies.
    Special techniques such as the feedback neutralization \cite{Akhtar2007ActiveNoiseControl, Akhtar2007AcousticFeedbackNeutralization} should be adopted to eliminate this side effect, otherwise it would impair the robustness of the system.
}

{
Accurate and computationally efficient prediction models are the fundamental requirement for audio applications involving PALs, such as ANC systems using PALs.
Therefore, two kinds of improved prediction models based on the partial-wave expansion method are proposed in Chap.~\ref{chap:predict_model}.
The first one is called the spherical wave expansion (SWE) solution, which simplifies the five-fold integral of the expression for audio sound into a three-fold summation without additional approximations and assumptions.
The second one is called the cylindrical wave expansion (CWE) solution, which gives a two-fold summation with the assumption that one dimension of the radiation surface is large enough when compared to the wavelength.
The key step for these two methods is to express the Green's function as the superposition of spherical and cylindrical waves.
Numerical simulations were used to validate the proposed solutions, and demonstrate that the proposed solution is much faster than existing models (e.g., GBE method).
Numerical results also show the proposed CWE solution improves agreement with experimental results in \cite{Shi2015ConvolutionModelComputing} when compared to existing models (e.g., convolution method).
The framework and results in Chap.~\ref{chap:predict_model} provide a more convenient and reliable tool to model a PAL in audio applications, such as calculating the secondary paths in Chap.~\ref{chap:anc}.
}

{
ANC and other audio systems are used in various kinds of acoustic environments, where the sound waves experience the reflection, transmission, scattering, and other physical phenomena. 
The theory for traditional dynamic loudspeakers has been well developed, so these physical phenomena can be appropriately modelled and investigated when they are used as secondary sources in ANC or other audio systems.
However, the physical generation of the audio beams from a PAL differs from the traditional loudspeakers, and their physical properties are still unclear.
Chapter \ref{chap:phys} proposed a full-wave based model to investigate the reflection from an infinitely large reflecting surface, the transmission through a thin partition, and the scattering by a sphere (simulating a human head).
The model has been validated by experiments conducted in anechoic rooms.
It is found a typical phenomenon is that the directivity of audio sound generated by a PAL is severely deteriorated if sound waves are reflected from a non-rigid surface, truncated by a thin partition, or scattered by a rigid sphere. 
The reason is that the directivity of audio sound is maintained by the ultrasound, which is more sensitive to the acoustic environment than the audio sound.
This implies the sharp directivity for PALs is not guaranteed as expected when they are used in complex acoustic environments.
Therefore, the directional performance would be deteriorated in ANC or other audio systems using PALs.
The methods and results presented in Chap.~\ref{chap:phys} provide a guidance for analyzing the effects of the reflection, transmission, and scattering on the performance of ANC or other audio systems using PALs.
}

% As the fast development of the commercial PAL products, they are used in many scenarios.
% Therefore, this thesis investigates the physical properties of audio sound generated by a PAL, which includes the reflection from an infinitely large reflecting surface, transmission through a thin partition, and the scattering by a sphere (simulating a human head). 
% Corresponding prediction models are proposed and experimental results are presented to validate the methods.
% It is found a typical phenomenon is that the directivity of audio sound generated by a PAL is severely deteriorated if sound waves are reflected from a non-rigid surface, truncated by a thin partition, or scattered by a rigid sphere. 
% The reason is that the directivity of audio sound is maintained by the ultrasound, which is more sensitive to the acoustic environment than audio sound.

{
    Finally, Chap.~\ref{chap:anc} investigated the applications of using PALs as secondary sources in ANC systems.
    The performance of a single channel  ANC system using one PAL was firstly explored in Sec.~\ref{sec:ancpal}.
    A laser Doppler vibrometer (LDV) was used in the system to sense the error signal at the error point remotely with less obstructions at the quiet zone. 
    The experiment results showed that such an ANC system can achieve similar overall noise reductions up to 6 kHz at the ear as a similar one albeit using a traditional loudspeaker.
    This work demonstrated the feasibility of the combining the remote sensing technique (e.g., LDV in this thesis, or remote microphone technique \cite{Das2011PerformanceEvaluationActive, Jung2018EstimationPressureListener}) and the PAL to cancel a broadband noise.
    In Sec.~\ref{sec:anpalqz}, the feasibility of using multiple PALs in a multi-channel ANC system was then investigated.
    Simple empirical formulae have been proposed to estimate the size of the quiet zone generated by PALs, which were validated by experimental results. 
    The quiet zone size generated by PALs is found to be similar to that observed with
    traditional omnidirectional loudspeakers. However, the spillover effects of using PALs as  secondary sources are much smaller than traditional loudspeakers, indicating that they can create  a larger quiet zone around the target point without affecting other areas.
    This is because PALs  are a highly directional loudspeakers, whereas traditional loudspeakers are omnidirectional. 
    Therefore, PALs provide promising alternative secondary sources in multi-channel ANC systems, such as a virtual sound barrier system \cite{Qiu2019IntroductionVirtualSound}.
% Finally, the ANC systems using PALs are designed and investigations on the feasibility of cancelling a broadband noise and creating a large quiet zone are conducted. 
% The noise reduction performance for each case is compared to that obtained using traditional omnidirectional loudspeakers.
% It is found the ANC system using PALs can achieve similar overall noise reductions up to 6 kHz at the human ear as a similar one albeit using a traditional omni-directional loudspeaker. 
% The size of the quiet zones generated by PALs is similar to that observed with traditional omnidirectional loudspeakers.
% However, the effects of using PALs on the sound field outside the target zone is much smaller
% due to their sharp radiation directivity. 
% Both single and multi channel ANC experiments are designed and conducted, and experimental results validate the findings.
}

\section{Future work}
Based on the findings in this thesis, the future work is identified as follows:

\begin{outline}
    \1 {
        The SWE method proposed in Sec.~\ref{Section41} is much faster than existing methods, which enables realiable and fast computations to model a PAL in ANC and other audio systems using PALs.
    However, it is applicable only for a circular PAL.
    It is worth exploring the accurate and computationally efficient prediction models  for a PAL that has other shapes, such as the rectangular PALs used to steer the audio beam \cite{Shi2013InvestigationSteerableParametric}, and the hexagonal PALs fabricated to realize a length-limited PAL \cite{Lam2014FeasibilityLengthlimitedParametric}.
}
    % It is worth investigating the SWE for a circular PAL that has asymmetric velocity profile, and for a PAL that has other shapes (e.g., rectangular).

    % \1 The SWE solution proposed given by Eq.~(\ref{eq:90jsfd12}) is a three-fold summation.
    % However, the numerical computation of an integral over triple spherical Bessel functions given by Eq.~(\ref{eq:spoi4e3fj203}) is required.
    % The simplification for this integral would result in faster calculation for the audio sound generated by a PAL.

    % \1 The CWE solution proposed in Sec.~\ref{sec:cwe_pal} requires that the radiation surface of the PAL array is infinitely long in one dimension. 
    % This requirement is easy to satisfy because the ultrasonic wavelength is usually much smaller than an ordinary PAL. 
    % However, this is not always the case for the audio sound and so prediction accuracy at low audio frequencies may deteriorate when using the cylindrical expansion and this remains to be addressed. 

    % \1 The simple plane wave reflection coefficient is used in the modelling of the reflection of audio sound generated by a PAL as described in Sec.~\ref{chap:phys}.
    % The reflection of intensive 
    % \1 It is has been demonstrated in Sec.~\ref{sec:phys_reflection} that the reflection of the audio sound generated by a PAL is profoundly, which decays slowly as propagating in air.
    % This property is annoying in audio applications. 
    % It is necessary to investigate the methods to confine the audio sound only in the near field of the PAL.

    \1 Although the scattering by a rigid sphere of audio sound generated by a PAL has been {investigated} in Sec.~\ref{sec:phys_scat},
    the behavior of audio sound when two or more spheres (simulating the human heads of multiple listeners in applications) exist is unclear, which requires further investigation.

    \1 Although the reflection from a surface, transmission through a thin partition, and scattering by a rigid sphere of audio sound generated by a PAL have been studied in Chap.~\ref{chap:phys}, it is worth investigating the propagation of audio sound in other acoustic environments, such as an enclosed cabin with reverberations.

    \1 
    {
        The audio sound generated by a PAL has a rate of 12 dB/octave decrease as the frequency is halved \cite{Yoneyama1983AudioSpotlightApplication}. 
        Therefore, the low frequency response of PALs is poor which may limit their applications. 
        % There are ways to improve the low frequency response, such as 
        Recently, a phononic crystal was proposed to improve the directivity of PALs \cite{Cervenka2021ParametricAcousticArray}. 
        It also shows more than 12 dB amplification is realized at low frequencies.
        Although the metamaterial usually achieves good performance in a narrowband range for traditional sound sources, it is not a disadvantage for PALs as a narrowband of ultrasonic frequency corresponds to a \revA{wideband} of audio frequencies.
        It is therefore worth exploring using other types of metamaterials to improve the low frequency response or manipulate the audio sound generated by the PAL. 
    }
    % The poor low frequency response of PALs may limit their use in real applications of ANC systems using PALs at low frequencies. 
    % Therefore, it is necessarily to investigate the methods in improving the low frequency response of PALs.
    % It is also a promising alternative to use a hybrid secondary source, i.e., the combination of traditional loudspeakers and PALs, in an ANC system.

    \1 In Sec.~\ref{sec:anpalqz}, all of the points inside the target zone  to be controlled are chosen here as error points, which requires many error sensors and a high-performance digital signal processor. 
    To reduce the number of error sensors, it is desirable to conduct further studies on the optimal error sensing strategy when using PALs. 
    
    \1 {It has been shown in Chap.~\ref{chap:phys} that the sharp directivity for PALs is not guaranteed as expected when they are used in complex acoustic environments. 
        Therefore, it is necessary to investigate how to avoid this side effect or improve the directivity of the audio beam when PALs are not used in a free field.
    }
    
    \1 {All analysis in this thesis is conducted in the frequency domain.
    In real applications, it requires the reproduction of a real time signal for the PAL.
    The main tool for predicting and analyzing the distortion performance of a PAL in the time domain is the Berktay's far field solution \cite{Berktay1965PossibleExploitationNonlinear}.
    Due to many approximations assumed in Berktay's solution, its prediction accuracy has been questioned \cite{Farias2015RayleighDistanceAbsorption, FariasAlvarez2018SoundPropagationNarrow}.
    Therefore, it is necessary to develop an accurate but computationally efficient model in the time domain.
}
    
    \1 {Many studies \cite{Boodoo2015ReviewEffectReflective, Tao2017PerformanceMultichannelActive, Zhong2019IncreasingPerformanceActive, Zhong2019IncreasingPerformanceActivea, Zhong2020PerformanceActiveNoise,  Sas1995ActiveControlSound, Tarabini2009ActiveControlNoisea, Zhang2017PerformanceSnoringNoise, Lin2004ActiveControlRadiation, Zou2008PerformanceAnalysisVirtual, Liu2018ActiveControlStrategy, Elliott2020ActiveControlSound} have demonstrated the physical properties (e.g., reflection, transmission, and scattering) of secondary sources have significant effects on the performance of ANC systems.
    These properties for audio sound generated by PALs have been systematically investigated in Chap.~\ref{chap:phys}.
    However, the effects of them on ANC systems remain to study.
}

\end{outline}
