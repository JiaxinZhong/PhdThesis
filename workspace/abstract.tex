\begin{abstract}
\addchaptertocentry{\abstractname} % Add the abstract to the table of contents
Parametric array loudspeakers (PALs) are known for their capability of generating highly directional audio sound waves.
Owing to this feature, they are used as secondary sources in active noise control (ANC) systems to mitigate the unwanted noise in the target regions whilst at the same time minimizing spillover effects on other areas. 
The primary aim of this thesis is to investigate the feasibility of using multiple PALs in an ANC system to create a large quiet zone.
To achieve this, a partial wave expansion model is proposed first based on the quasilinear solution of both Westervelt and Kuznetsov equations to predict the audio sound generated by a PAL in a free field.
The model is then extended to accommodate reflection, transmission, and scattering phenomena, which are common in real applications and can have significant effects on the noise reduction performance of ANC systems.
The proposed model is validated by experiments conducted in anechoic rooms, and the validated model incorporated with the multi-channel ANC theory is then used to investigate the quiet zone size controlled by multiple PALs.

    It is found the existing prediction models for PALs are either inaccurate or time-consuming, while the proposed model is more than 100 times faster in both near and far fields without any loss of accuracy.
It therefore enables reliable and fast simulations for multi-channel ANC systems, which require heavy computations due to large numbers of PALs. 
A key finding is that the directivity of the audio sound generated by a PAL is severely deteriorated if sound waves are reflected from a non-rigid surface, truncated by a thin partition, or scattered by a sphere (simulating a human head). 
This implies the sharp directivity for PALs is not guaranteed as expected when they are used in complex acoustic environments.
Finally, both simulations and experiments showed that multiple PALs can create a large quiet zone of comparable size when compared to traditional omnidirectional loudspeakers. 
However, the spillover effects of using PALs on the sound field outside the quiet zone are much smaller, which demonstrates PALs provide a promising alternative as secondary sources in multi-channel ANC systems.


\end{abstract}
